\documentclass[a4paper,10pt]{article}

%mudar a definição abaixo para alterar o tipo de currículo
%valores possíveis:
%d = developer
%p = professor
\newcommand{\cvtype}{d}

%A Few Useful Packages
\usepackage{marvosym}
\usepackage{fontspec} 					%for loading fonts
\usepackage{xunicode,xltxtra,url,parskip} 	%other packages for formatting
\RequirePackage{color,graphicx}
\usepackage[usenames,dvipsnames]{xcolor}
\usepackage[big]{layaureo} 				%better formatting of the A4 page
% an alternative to Layaureo can be ** \usepackage{fullpage} **
\usepackage{supertabular} 				%for Grades
\usepackage{titlesec}					%custom \section

%Setup hyperref package, and colours for links
\usepackage{hyperref}
\definecolor{linkcolour}{rgb}{0,0.2,0.6}
\hypersetup{colorlinks,breaklinks,urlcolor=linkcolour, linkcolor=linkcolour}

%FONTS
\defaultfontfeatures{Mapping=tex-text}
%TODO \setmainfont[SmallCapsFont = Fontin SmallCaps]{Fontin}
%%% modified for Karol Kozioł for ShareLaTeX use
%\setmainfont[
%SmallCapsFont = Fontin-SmallCaps.otf,
%BoldFont = Fontin-Bold.otf,
%ItalicFont = Fontin-Italic.otf
%]
%{Fontin.otf} 
%%%

%CV Sections inspired by: 
%http://stefano.italians.nl/archives/26
\titleformat{\section}{\Large\scshape\raggedright}{}{0em}{}[\titlerule]
\titlespacing{\section}{0pt}{3pt}{3pt}
%Tweak a bit the top margin
%\addtolength{\voffset}{-1.3cm}

%Italian hyphenation for the word: ''corporations''
\hyphenation{im-pre-se}

%-------------WATERMARK TEST [**not part of a CV**]---------------
\usepackage[absolute]{textpos}

\setlength{\TPHorizModule}{30mm}
\setlength{\TPVertModule}{\TPHorizModule}
\textblockorigin{2mm}{0.65\paperheight}
\setlength{\parindent}{0pt}

%--------------------BEGIN DOCUMENT----------------------
\begin{document}

%WATERMARK TEST [**not part of a CV**]---------------
%\font\wm=''Baskerville:color=787878'' at 8pt
%\font\wmweb=''Baskerville:color=FF1493'' at 8pt
%{\wm 
%	\begin{textblock}{1}(0,0)
%		\rotatebox{-90}{\parbox{500mm}{
%			Typeset by Alessandro Plasmati with \XeTeX\  \today\ for 
%			{\wmweb \href{http://www.aleplasmati.comuv.com}{aleplasmati.comuv.com}}
%		}
%	}
%	\end{textblock}
%}

\pagestyle{empty} % non-numbered pages

\font\fb=''[cmr10]'' %for use with \LaTeX command

%--------------------TITLE-------------
\par{\centering
		{\Huge Luiz Guilherme Chagas Moraes \textsc{Jardim}
	}\bigskip\par}

%--------------------SECTIONS-----------------------------------
%Section: Personal Data
\section{Personal Data}

\begin{tabular}{rl}
    \textsc{Place and Date of Birth:} & Rio de Janeiro, RJ  | November 8th, 1989 \\
    \textsc{Address:}   & Rua Chãos Monte, 5, R/C, Lijó\\& Vila Nova de Gaia, Porto, Portugal \\
    \textsc{Phone:}     & +351 937 244 326\\
    \textsc{email:}     & \href{mailto:luizchagasjardim@gmail.com}{luizchagasjardim@gmail.com} \\
\end{tabular}

\section{Presentation}
Dear recruiters,

\if \cvtype d
%developer
{\quad}I've been working as a developer for over eight years, programming mainly in \textbf{C++}, but also \textbf{Python}, \textbf{Rust} and \textbf{Java}, among others. I also have experience in other technologies, leadership roles and teaching. What I enjoy the most in programming is documenting, writing unit tests, refactoring legacy code and sharing knowledge. My extensive background in Mathematics, Science and Education sometimes comes in handy.

{\quad}My goal in life is to learn and teach as much as possible. Software development is a field where there are many multi disciplinary projects with a lot different subjects and technologies to learn about. I also work on personal projects when I have the time.

{\quad}\footnotesize{For more details, please go to \href{http://luizchagasjardim.github.io}{luizchagasjardim.github.io}}
\else
	\if \cvtype p 
%professor

	\fi
\fi

%Section: Work Experience at the top
\section{Work Experience}
\begin{tabular}{r|p{11cm}}

\textsc{January 2022} & Developer and Scrum Master at \textsc{Velocix}\\
\textsc{Current} & Developer in VxOA, Velocix Origin Application, which is responsible for ingesting, encrypting/decrypting and sending video assets using various streaming protocols.\\

\multicolumn{2}{c}{}\\


\textsc{January 2020} & Developer and Scrum Master at \textsc{Critical Techworks}\\
\textsc{January 2022} & Development of software used for testing and quality assurance of BMW embedded software, such as a test framework written in C++ and a CI website with Java backend. After a year or so I also started acting as the Scrum Master for my team. We use technologies such as GTest, Git, SVN, Postgresql, Javascript, HTML, CSS, Docker, Saltstack and Openstack. I also take part in many company activities, mainly the Rust Community of Practice (a group co-created by me to study and talk about Rust, with weekly sessions) and the C++ daily (a 15 min meeting open to the whole company where we talk about anything related to C++).\\

\multicolumn{2}{c}{}\\


\textsc{June 2019} & Consultant at \textsc{Natixis} through \textsc{Altran}\\
\textsc{December 2019} & I worked on a financial software called Summit. The software was written in C++, but there were also many scripts in Python. We used technologies such as GTest, XLDeploy, Git, Jenkins, Bitbucket and Jira. I was also one of the leaders of the Craftsman Academy, a weekly session that consisted of a short presentation followed by mob programming.\\

\multicolumn{2}{c}{}\\

\textsc{March 2016} & Systems analyst at \textsc{Brazilian Navy Research Institute}\\
\textsc{March 2019} & Development of a ship's engine room simulator for the Mercantile Navy\\&
\footnotesize{Responsible for the control module (C++), mantaining the mathematical models (C++/Fortran) and communication between these components and the graphical interface (Java);}\\&
\footnotesize{Also reponsible for the test module (C++/Java), which uses a .json file to simulate user operations and verifies the outcomes given by the models. It also checked the validity of \textsc{fxml}, \textsc{xml} and the database (Postgres). These tests were run using Gitlab's CI.}\\&
\footnotesize{Modifications on the database (pgSQL), graphical interfaces (JavaFx) and model analysis (\textsc{xml}, Simulink).}\\

\multicolumn{2}{c}{}\\

\textsc{Before March 2016} & I used to teach math\\
& \footnotesize{For more details, please go to \href{http://luizchagasjardim.github.io}{luizchagasjardim.github.io}}
\end{tabular}

%Section: Education
\section{Education}
\begin{tabular}{rl}

\textsc{December} 2016 & \textsc{PhD in Mechanical Engineering}\\&
\textbf{Universidade do Estado do Rio de Janeiro}, \textsc{PPGEM}\\&
Research Area: Transport Phenomena\\&
Thesis: “A hyperbolic model for saturated-unsaturated transition\\& simulation in porous media”\\&
Advisor: Rogério Martins Saldanha da Gama, PhD.\\&
Co-Advisor: José Julio Pedrosa Filho, PhD.\\
\\
\textsc{July 2014} & \textsc{MSc in Mechanical Engineering}\\&
\textbf{Universidade do Estado do Rio de Janeiro}, \textsc{PPGEM}\\&
Research Area: Transport Phenomena\\&
Thesis: “A new description of mass tranfer in porous media with\\&
saturated-unsaturated transition”\\&
Advisor: Rogério Martins Saldanha da Gama, PhD.\\&
Co-Advisor: José Julio Pedrosa Filho, PhD.\\
\\
December 2012 & \textsc{Undergraduate Degree in Mathematics}\\&
\textbf{Universidade do Estado do Rio de Janeiro}, \textsc{IME}\\&
Thesis: “Cauchy Stress Tensor”\\&
Advisor: José Julio Pedrosa Filho, PhD.\\

\end{tabular}

%Section: Languages
\section{Languages}
\begin{tabular}{rl}
\textsc{Brazilian Portuguese:}&Native\\
\textsc{English:}&Advanced\\
\end{tabular}

%\newpage
%\hypertarget{gmat}{\textsc{Gmat}\setmainfont{LMRoman10 Regular}\textregistered\setmainfont[SmallCapsFont=Fontin-SmallCaps]{Fontin-Regular}}

%\XeTeXpdffile ''GMAT.pdf'' page 1 scaled 800

\end{document}

